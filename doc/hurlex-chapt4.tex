% -*- coding: UTF-8 -*-
% hurlex-chapt4.tex
% hurlex 开发文档 第4章内容

\section {相关库函数和调试打印函数}

\par 本来计划着在这一章就开始介绍屏幕显示,但是考虑到我们还缺少必须的"基础设施"代码,比如我们需要的一些库函数和调试使用的函数\allowbreak
都还处于未完成状态,所以我们还是先完成这些基础代码再继续吧。

\par 不必沮丧我们一时的滞后。我们常说磨刀不误砍柴工,这些辅助函数会在后期的开发中给我们带来很多的方便。另外我们还需要介绍\allowbreak
调试的方法和一些必须的调试脚本,这样的话倘若代码有问题我们就可以很轻松的排错了。

\par 首先是几个端口读写函数的实现。我们之前介绍过端口地址的概念,有的外设的相关存储单元在端口地址里,需要CPU通过特殊的in/out\allowbreak
指令才可以访问到。但是C语言并没有直接操作端口的方法,而且频繁的内联汇编麻烦又容易出错。所以我们索性定义几个端口读写函数。\allowbreak
代码如下:

\begin{lstlisting}[language = C, label = , caption = init/entry.c]

\end{lstlisting}


